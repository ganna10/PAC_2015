\begin{frame}
    \frametitle{Outline} 
    \tableofcontents[currentsection]
\end{frame} 

\begin{frame}
    \frametitle{Main Research Question} 

    \vspace{-0.6cm}
    \begin{center}
        \Large{How do the simplification techniques \\used in different chemical mechanisms \\affect \ce{O_x} production?}
    \end{center}
\end{frame}

\begin{frame}
    \frametitle{Chemical Mechanisms}

    \vspace{-0.4cm}
    {
        \setstretch{1.17}
        \begin{table}%[!ht]
            \begin{center}
                \small\makebox[\textwidth][c]{%
                \begin{tabular}{llP{5.2cm}}
                    \toprule
                    \textbf{Chemical} & \multirow{2}{*}{\textbf{Lumping Approach}}  & \multirow{2}{*}{\textbf{Reference}} \\ \textbf{Mechanism} & & \\ \toprule
                    MCM v3.2 & No lumping & [\url{http://mcm.leeds.ac.uk/MCM/}] \\ \hline
                    \multirow{2}{*}{MCM v3.1} & \multirow{2}{*}{No lumping} & [Saunders et al., ACP, 2003] \\ & & [Jenkin et al., ACP, 2003] \\ \hline
                    CRI v2 & Lumped intermediates & [Jenkin et al., AE, 2008]\\ \hline
                    MOZART-4 & Lumped molecule & [Emmons et al., GMD, 2010] \\ \hline
                    RADM2 & Lumped molecule & [Stockwell et al., JGR, 1990] \\ \hline
                    RACM & Lumped molecule & [Stockwell et al., JGR, 1997] \\ \hline
                    RACM2 & Lumped molecule & [Goliff et al., AE, 2013] \\ \hline
                    CBM-IV & Lumped structure & [Gery et al., JGR, 1989] \\ \hline
                    CB05 & Lumped structure & [Yarwood et al., EPA report, 2005] \\ \bottomrule
            \end{tabular}}
            \end{center}
        \end{table}
    }
\end{frame}

\begin{frame}
    \frametitle{Boxmodel Setup}

    \vspace{-0.5cm}
    \begin{itemize}
        \item MECCA boxmodel over 7 days. \vspace{3mm}
        \item Initial NMVOC typical of Los Angeles. \vspace{3mm}
        \item Same NMVOC emissions and reactive carbon \\in each model run. \vspace{3mm}
        \item NO source tuned for maximum \ce{O3} production. \vspace{3mm}
        \item Mechanisms tagged for each NMVOC.
    \end{itemize}
\end{frame}

\begin{frame}
    \frametitle{Organic Degradation Product Tagging}
    \vspace{-1.0cm}
    \begin{center}
        \input{tex_files/tagging_diagram}
    \end{center}
\end{frame}

{
    \usebackgroundtemplate{%
        \vbox to \paperheight{\vfil\hbox to \paperwidth{\hfil\includegraphics[height=0.95\paperheight, width = 0.95\paperwidth]{../Plotting_scripts/MCMv3_2_tagged_non_tagged_Ox_budget}\hfil}\vfil}
    }
    \begin{frame}[plain]
    \end{frame}
}

\begin{frame}
    \frametitle{TOPP Calculation}

    \begin{itemize}
        \item Attribute daily \ce{O_x} production to each NMVOC. \vspace{5mm}
        \item Sum daily \ce{O_x} production from each NMVOC. \vspace{5mm}
        \item Normalise by total emissions of the NMVOC on day 1. \vspace{5mm}
    \end{itemize}
\end{frame}

{
    \usebackgroundtemplate{%
        \vbox to \paperheight{\vfil\hbox to \paperwidth{\hfil\includegraphics[height=0.95\paperheight, width = 0.95\paperwidth]{../Plotting_scripts/O3_mixing_ratios_mech_comp}\hfil}\vfil}
    }
    \begin{frame}[plain]
    \end{frame}
}

{
    \usebackgroundtemplate{%
        \vbox to \paperheight{\vfil\hbox to \paperwidth{\hfil\includegraphics[height=0.95\paperheight, width = 0.95\paperwidth]{../Plotting_scripts/TOPP_daily_time_series_pentane_toluene}\hfil}\vfil}
    }
    \begin{frame}[plain]
    \end{frame}
}

{
    \usebackgroundtemplate{%
        \vbox to \paperheight{\vfil\hbox to \paperwidth{\hfil\includegraphics[height=0.95\paperheight, width = 0.95\paperwidth]{../Plotting_scripts/pentane_Ox_production_by_C_number}\hfil}\vfil}
    }
    \begin{frame}[plain]
    \end{frame}
}

{
    \usebackgroundtemplate{%
        \vbox to \paperheight{\vfil\hbox to \paperwidth{\hfil\includegraphics[height=0.95\paperheight, width = 0.95\paperwidth]{../Plotting_scripts/toluene_Ox_production_by_C_number}\hfil}\vfil}
    }
    \begin{frame}[plain]
    \end{frame}
}

{
    \usebackgroundtemplate{%
        \vbox to \paperheight{\vfil\hbox to \paperwidth{\hfil\includegraphics[height=0.95\paperheight, width = 0.95\paperwidth]{../Plotting_scripts/net_reactive_carbon_loss_pentane_toluene}\hfil}\vfil}
    }
    \begin{frame}[plain]
    \end{frame}
}

{
    \usebackgroundtemplate{%
        \vbox to \paperheight{\vfil\hbox to \paperwidth{\hfil\includegraphics[height=0.95\paperheight, width = 0.95\paperwidth]{../Plotting_scripts/HC5P_NO_C_numbers}\hfil}\vfil}
    }
    \begin{frame}[plain]
    \end{frame}
}

\begin{frame}
    \frametitle{Conclusions}
    \begin{columns}[onlytextwidth]
        \begin{column}{0.5\textwidth}
            \vspace{-4mm}
            \hspace{-9.8mm}
            \includegraphics[width=1.13\textwidth]{../Plotting_scripts/O3_mixing_ratios_mech_comp}
        \end{column}%
        \begin{column}{0.5\textwidth} 
            \vspace{-4mm}
            \begin{itemize}
                \item Reduced mechanisms break down many VOC faster than MCM. \vspace{4mm}
                \item Many VOC produce similar \ce{Ox} to MCM on first day, but not subsequent days. 
            \end{itemize}
        \end{column}
    \end{columns}
\end{frame}

\begin{frame}
    \frametitle{Paper Status}
    \begin{itemize}
        \item Advanced draft of paper sent for internal review. \vspace{1cm}
        \item Discuss draft as part of this meeting.
    \end{itemize}
\end{frame}
